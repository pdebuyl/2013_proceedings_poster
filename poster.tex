\documentclass[portrait]{sciposter}

\usepackage{amsmath}
\usepackage{amssymb}
\usepackage{multicol}
\usepackage{graphicx}
\usepackage{url}
\usepackage[utf8]{inputenc}
\usepackage{tikz}


% for xdvi:  -paper 841mmx1189mm
% dvips -T 841mm,1189mm -o poster.ps poster.dvi

\definecolor{lightgrey}{rgb}{0.940,0.940,0.940}
% uncomment for grey background to \section boxes 
% for use with default option boxedsections

%\definecolor{BoxCol}{rgb}{0.9,0.9,1}
% uncomment for light blue background to \section boxes 
% for use with default option boxedsections
\usepackage{xkeyval}

\makeatletter

\define@key{mybox}{corners}{\def\my@corners{#1}}
\define@key{mybox}{color}{\def\bg@color{#1}}
\define@key{mybox}{width}{\setlength\mybox@width{#1}}
\define@key{mybox}{position}{\def\box@position{#1}}
\define@key{mybox}{innerposition}{\def\box@innerposition{#1}}
\define@key{mybox}{opacity}{\def\box@opacity{#1}}
\define@key{mybox}{height}{\def\box@height{#1}}
\newbox\my@tempbox
\newlength\my@height
\newlength\my@width
\newlength\mybox@width
\newlength\innerbox@width
\newlength\box@heightlgth

\newcommand{\mybox}[2][]{%
  \setlength\mybox@width{\linewidth}%
  \def\box@position{T}%
  \def\box@innerposition{s}%
  \def\box@opacity{0.75}%
  \def\bg@color{gray!25}%
  \def\my@corners{2mm}%
  \let\box@height\@empty
  \setkeys{mybox}{#1}%
  \setlength\innerbox@width{\mybox@width}%
  \advance\innerbox@width by -2mm%
  \setbox\my@tempbox=\hbox{%
  \ifx\box@height\@empty%
    \begin{minipage}{\innerbox@width}%
  \else%
    \setlength\box@heightlgth{\box@height}%
    \advance\box@heightlgth by -\my@corners%
    \advance\box@heightlgth by -1.7ex%
    \begin{minipage}[\box@position][\box@heightlgth][\box@innerposition]%
	    {\innerbox@width}%
  \fi%
      #2%
    \end{minipage}%
  }%
  \my@height=\ht\my@tempbox%
  \my@width=\wd\my@tempbox%
  %
  \advance\my@width by 2mm%
  %
  \begin{minipage}[\box@position]{\mybox@width}%
  \begin{tikzpicture}[rounded corners=\my@corners, fill=\bg@color]
      \fill[fill opacity=\box@opacity, scale=0.5]
      	(-\my@width,-2\my@height) rectangle (\my@width, 2\my@height);
      \path (0,0) node {\box\my@tempbox} ;
  \end{tikzpicture}%
  \end{minipage}%
}

\definecolor{BoxCol}{rgb}{1,1,1}
% uncomment for dark blue \section text 


\title{EuroSciPy 2013 Proceedings}

% Note: only give author names, not institute
\author{}
 
% insert correct institute name
\institute{}

 % shows author email address below institute


%\date is unused by the current \maketitle

% The following commands can be used to alter the default logo settings
%\noleftlogo
\leftlogo[1.2]{euroscipy_logo}  % defines logo to left of title (with scale factor)
\rightlogo[1.2]{euroscipy_logo}  % same but on right
%\norightlogo
% NOTE: This will require presence of files logoWenI.png and RuGlogo.png, 
% or other supported format in the current directory  
%%%%%%%%%%%%%%%%%%%%%%%%%%%%%%%%%%%%%%%%%%%%%%%%%%%%%%%%%%%%%%%%%%%%%%%%%%%%%%%%
%%% Begin of Document

\begin{document}

%define conference poster is presented at (appears as footer)
\conference{{\large EuroSciPy 2013}}

\maketitle

%%% Begin of Multicols-Enviroment
\begin{multicols}{2}

If you have a github account, go to github.com and register.\\

Go to \url{https://github.com/euroscipy/euroscipy_proceedings} and click the
fork button.

  \begin{figure}[h]
    \centering
    \includegraphics[width=\linewidth]{fork_draw}
  \end{figure}

  You should now be on your {\em own} account and your {\em own} fork of
  euroscipy\_proceedings.

  \begin{figure}[h]
    \centering
    \includegraphics[width=0.7\linewidth]{myfork_draw}
  \end{figure}

  You may now clone (i.e. get the code on your computer) the repository, from
  your {\em own} version of it. Replace pdebuyl by your github login.

  \begin{figure}[h]
    \centering
    \includegraphics[width=\linewidth]{txt}
  \end{figure}

  To add your contribution, create a directory in ``papers'' with your
  name. Using the paper ``00\_vanderwalt.rst'' as a template, add and edit your
  contribution.\\

  You should then {\tt git add} the file and {\tt git commit} to register your
  to register your
  addition. Figures should be added as well. Edit and commit until the paper is
  ready. The script {\tt make\_paper.sh} can be used to produce a pdf file in
  ``output/your\_name/paper.pdf''.

  Once you are satisfied, {\tt git push} your contribution. It now lies within
  {\em your own} github account.

  \begin{figure}[h]
    \centering
    \includegraphics[width=.6\linewidth]{txt_push}
  \end{figure}

  Your version has a ``pull request'' list, click on that.

  \begin{figure}[h]
    \centering
    \includegraphics[width=\linewidth]{mycommit}
  \end{figure}

  In the pull request list, there is a ``new Pull Request'' button.

  \begin{figure}[h]
    \centering
    \includegraphics[width=0.75\linewidth]{mypr}
  \end{figure}

  The ``pull request'' is from your account to the euroscipy account, create it.

  \begin{figure}[h]
    \centering
    \includegraphics[width=0.6\linewidth]{mypr_create}
  \end{figure}


  Once you have submitted a pull request, the refereeing will start. Refereeing
  will be done on the pull request and will be public.

\end{multicols}

\begin{multicols}{2}
  As a reminder, the requirements are:


  \begin{itemize}
  \item IEEETran (often packaged as ``texlive-publishers'', or download from
   CTAN \url{http://www.ctan.org/tex-archive/macros/latex/contrib/IEEEtran/}
  \item AMSmath LaTeX classes (included in most LaTeX distributions)
  \item ``docutils'' 0.8 or later (``easy\_install docutils'')
  \item ``pygments'' for code highlighting (``easy\_install pygments'')
  \end{itemize}

If you need support, please contact Nelle Varoquaux \url{nelle.varoquaux@gmail.com} or Pierre de Buyl \url{pdebuyl@ulb.ac.be}

\end{multicols}

\vspace*{5em}
\begin{center}
\begin{minipage}{.5\linewidth}
\section{Summary}
\vspace*{-0.9em}
\mybox[corners=0pt, color=lightgrey, width=\linewidth]{
\begin{itemize}
\item Fork \url{https://github.com/euroscipy/euroscipy\_proceedings}   
\item Clone the fork: 
  \begin{itemize}
    \item \texttt{git clone https://github.com/euroscipy/euroscipy\_proceedings}

  \end{itemize}
\item Write your paper in \texttt{papers/00\_lastname.rst}
\item Commit and push it:
  \begin{itemize}
  \item \texttt{git add papers/00\_lastname.rst}
  \item \texttt{git commit -m "My paper"}
  \item \texttt{git push origin master}
  \end{itemize}

\item Create the pull request and wait for feedback.
\end{itemize}

}
\end{minipage}
\end{center}

\end{document}

