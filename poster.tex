\documentclass[portrait]{sciposter}

\usepackage{amsmath}
\usepackage{amssymb}
\usepackage{multicol}
\usepackage{graphicx}
\usepackage{url}
\usepackage[utf8]{inputenc}

% for xdvi:  -paper 841mmx1189mm
% dvips -T 841mm,1189mm -o poster.ps poster.dvi

%\definecolor{BoxCol}{rgb}{0.9,0.9,0.9}
% uncomment for grey background to \section boxes 
% for use with default option boxedsections

%\definecolor{BoxCol}{rgb}{0.9,0.9,1}
% uncomment for light blue background to \section boxes 
% for use with default option boxedsections

\definecolor{BoxCol}{rgb}{1,1,1}
% uncomment for dark blue \section text 


\title{EuroSciPy 2013 Proceedings}

% Note: only give author names, not institute
\author{}
 
% insert correct institute name
\institute{}

 % shows author email address below institute


%\date is unused by the current \maketitle

% The following commands can be used to alter the default logo settings
%\noleftlogo
\leftlogo[1.2]{euroscipy_logo}  % defines logo to left of title (with scale factor)
\rightlogo[1.2]{euroscipy_logo}  % same but on right
%\norightlogo
% NOTE: This will require presence of files logoWenI.png and RuGlogo.png, 
% or other supported format in the current directory  
%%%%%%%%%%%%%%%%%%%%%%%%%%%%%%%%%%%%%%%%%%%%%%%%%%%%%%%%%%%%%%%%%%%%%%%%%%%%%%%%
%%% Begin of Document

\begin{document}

%define conference poster is presented at (appears as footer)
\conference{{\large EuroSciPy 2013}}

\maketitle

%%% Begin of Multicols-Enviroment
\begin{multicols}{2}

If you have a github account, go to github.com and register.\\

Go to \url{https://github.com/euroscipy/euroscipy_proceedings} and click the
fork button.

  \begin{figure}[h]
    \centering
    \includegraphics[width=\linewidth]{fork_draw}
  \end{figure}

  You should now be on your {\em own} account and your {\em own} fork of
  euroscipy\_proceedings.

  \begin{figure}[h]
    \centering
    \includegraphics[width=0.7\linewidth]{myfork_draw}
  \end{figure}

  You may now clone (i.e. get the code on your computer) the repository, from
  your {\em own} version of it. Replace pdebuyl by your github login.

  \begin{figure}[h]
    \centering
    \includegraphics[width=\linewidth]{txt}
  \end{figure}

  To add your contribution, create a directory in ``papers'' with your
  name. Using the paper ``00\_vanderwalt.rst'' as a template, add and edit your
  contribution.\\

  You should then ``git add'' the file and ``git commit'' to register your
  addition. Figures should be added as well. Edit and commit until the paper is
  ready. The script ``make\_paper.sh'' can be used to produce a pdf file in
  ``output/your\_name/paper.pdf''.

  Once you are satisfied, ``git push'' your contribution. It now lies within
  {\em your own} github account.

  \begin{figure}[h]
    \centering
    \includegraphics[width=.6\linewidth]{txt_push}
  \end{figure}

  Your version has a ``pull request'' list, click on that.

  \begin{figure}[h]
    \centering
    \includegraphics[width=\linewidth]{mycommit}
  \end{figure}

  In the pull request list, there is a ``new Pull Request'' button.

  \begin{figure}[h]
    \centering
    \includegraphics[width=0.75\linewidth]{mypr}
  \end{figure}

  The ``pull request'' is from your account to the euroscipy account, create it.

  \begin{figure}[h]
    \centering
    \includegraphics[width=0.6\linewidth]{mypr_create}
  \end{figure}


  Once you have submitted a pull request, the refereeing will start. Refereeing
  will be done on the pull request and will be public.

\end{multicols}

\begin{multicols}{2}
  As a reminder, the requirements are:


  \begin{itemize}
  \item IEEETran (often packaged as ``texlive-publishers'', or download from
   CTAN \url{http://www.ctan.org/tex-archive/macros/latex/contrib/IEEEtran/}
  \item AMSmath LaTeX classes (included in most LaTeX distributions)
  \item ``docutils'' 0.8 or later (``easy\_install docutils'')
  \item ``pygments'' for code highlighting (``easy\_install pygments'')
  \end{itemize}

If you need support, please contact Nelle Varoquaux \url{nelle.varoquaux@gmail.com} or Pierre de Buyl \url{pdebuyl@ulb.ac.be}

\end{multicols}

\end{document}

